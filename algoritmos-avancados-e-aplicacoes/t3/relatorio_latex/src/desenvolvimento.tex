\section{Desenvolvimento}
	
	\subsection{Formulação}

Para desenvolver o algoritmo, foram utilizados os subsídios 1 e 2 dados na descrição do trabalho. A partir do subsídio 1, foi obtida a seguinte fórmula:

\[
	\begin{bmatrix}
		F_{n}\\
		F_{n-1}
	\end{bmatrix} = A^{n-1} \times B
	\text{, onde }
	A =
	\begin{bmatrix}
		1 & 1\\
		1 & 0
	\end{bmatrix}
	\text{ e }
	B = 
	\begin{bmatrix}
		F_{1}\\
		F_{0}
	\end{bmatrix}
	=
	\begin{bmatrix}
		1\\
		1
	\end{bmatrix}
\]

O algoritmo para a fórmula acima tem a complexidade $O(log_{2}n)$, uma vez que, pelo subsídio 2, a fórmula para calcular a $n$-ésima potência de uma matriz $A$ é:

\[
	\left \{
		\begin{array}{lll}
			pow(A, n) = A\text{, se } n = 1\\
			pow(A, n) = pow^{2}(A, n/2)\text{, se } n \text{ é par}\\
			pow(A, n) = A \times pow^{2}(A, \lfloor n/2 \rfloor)\text{, se } n \text{ é ímpar}
		\end{array}
	\right .
\]

A justificativa da complexidade $O(log_{2}n)$ vem do método de divisão e conquista aplicado na recorrência acima. Isso pode ser concluído observando-se que:

\begin{itemize}
	\item A cada chamada recursiva, o tamanho do problema é dividido em dois gerando uma árvore de recursão de altura $log_{2}n$.
	\item Em cada nível $k$ da árvore é realizada uma operação de multiplicação de matrizes considerada de complexidade $O(1)$, pois o resultado de $pow^{2}(A, \lfloor n/2^{k} \rfloor)$ é feito atribuindo-se $R \gets pow(A , \lfloor n/2^{k} \rfloor)$ e então $R \gets R \times R$ se $n$ é par, ou $R \gets A \times (R \times R)$ se $n$ é ímpar.
	\item Portanto: se em cada nível da árvore é feita uma operação $O(1)$ e sua altura é $log_{2}n$, a ordem complexidade é $O(1) \cdot log_{2}n \Rightarrow O(log_{2}n)$.
\end{itemize}

	\subsection{Implementação}
		
A implementação do projeto foi feita em linguagem C. Não houve a necessidade de criação de estruturas de dados, pois foram utilizadas somente matrizes que são simples arranjos de números inteiros em linguagem C.

Foi necessário usar variáveis inteiras de 64 bits para guardar os valores das operações, pois o próprio limite da entrada, $10^{11}$, já estoura o limite de representação de inteiros com 32 bits. Além disso, no pior caso, mesmo aplicando mod $10^{6}$ nas multiplicações, teríamos por algum instante um valor na ordem de $10^{12}$.

O código é composto de 3 funções que juntas executam o algoritmo. São elas:
\begin{verbatim}
unsigned long long int **matrix_mul(unsigned long long int **A,
      unsigned long long int **B, unsigned long long int mod);

unsigned long long int **matrix_pow(unsigned long long int **M,
      unsigned long long int n, unsigned long long int mod);

unsigned long long int F(unsigned long long int n,
      unsigned long long int mod);
\end{verbatim}

Confira a documentação no código-fonte do projeto.
