\section{Resultados}
Executando o programa para os casos de testes descritos nos itens do trabalho e tomando $\varepsilon = 0.000001$, $itmax = 1000$ e $\mathbf{x}^{(0)} = \mathbf{0}$, foram obtidas as seguintes soluções:

\begin{itemize}
	\item Caso de teste com $b_{i} = \sum_{j=1}^{n}a_{ij} \text{ para } i = 1, 2, \dots, n$ com solução esperada $x_{i} = 1 \text{ para } i = 1, 2, \dots, n$

	\begin{itemize}
		\item para $n = 50$\\
		\texttt{Quantidade de iteracoes: 557\\
			Vetor solucao do sistema:\\
			x = (  0.999993  0.999990  0.999988  0.999984  0.999980  0.999977  0.999974  0.999971  0.999968  0.999965  0.999963  0.999960  0.999958  0.999956  0.999954  0.999952  0.999951  0.999949  0.999948  0.999947  0.999946  0.999946  0.999945  0.999945  0.999945  0.999945  0.999946  0.999946  0.999947  0.999948  0.999949  0.999951  0.999952  0.999954  0.999956  0.999957  0.999960  0.999962  0.999964  0.999967  0.999969  0.999972  0.999974  0.999977  0.999980  0.999983  0.999986  0.999989  0.999992  0.999994  )}
		\item para $n = 100$\\
		\texttt{Quantidade de iteracoes: 1000\\
			Vetor solucao do sistema:\\
			x = (  0.999317  0.998988  0.998715  0.998294  0.997934  0.997598  0.997234  0.996881  0.996539  0.996194  0.995855  0.995521  0.995192  0.994867  0.994549  0.994236  0.993929  0.993628  0.993334  0.993046  0.992766  0.992493  0.992228  0.991971  0.991721  0.991480  0.991248  0.991024  0.990809  0.990603  0.990406  0.990219  0.990041  0.989873  0.989715  0.989567  0.989429  0.989301  0.989184  0.989077  0.988980  0.988894  0.988819  0.988754  0.988700  0.988657  0.988625  0.988603  0.988593  0.988593  0.988603  0.988625  0.988657  0.988700  0.988753  0.988817  0.988891  0.988976  0.989071  0.989176  0.989291  0.989416  0.989551  0.989695  0.989849  0.990012  0.990184  0.990365  0.990554  0.990753  0.990959  0.991174  0.991397  0.991627  0.991865  0.992110  0.992362  0.992621  0.992887  0.993158  0.993435  0.993719  0.994007  0.994301  0.994599  0.994902  0.995209  0.995520  0.995835  0.996153  0.996474  0.996800  0.997122  0.997452  0.997792  0.998104  0.998438  0.998825  0.999076  0.999378  )}
	\end{itemize}

	Analisando os valores numéricos obtidos, conclui-se que o programa converge para a resposta esperada $x_{i} = 1 \text{ para } i = 1, 2, \dots, n$

	\item Caso de teste com $b_{i} = 1/i \text{ para } i = 1, 2, \dots, n$
	
	\begin{itemize}
		\item para $n = 40$\\
		\texttt{Quantidade de iteracoes: 389\\
			Vetor solucao do sistema:\\
			x = (  0.748938  0.852286  0.911644  1.143467  1.248563  1.317490  1.414722  1.481012  1.528366  1.574063  1.607399  1.629788  1.645767  1.653826  1.654288  1.648471  1.636424  1.618402  1.594907  1.566175  1.532434  1.493949  1.450937  1.403592  1.352066  1.296556  1.237267  1.174168  1.107492  1.037716  0.964136  0.887048  0.808637  0.725354  0.636676  0.554129  0.462555  0.351604  0.280870  0.192106  )}
	\end{itemize}
\end{itemize}

