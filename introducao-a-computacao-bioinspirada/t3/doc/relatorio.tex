% =========================================================
%  Cabeçalho
% =========================================================

% Define papel, tamanho global de fonte, tipo de documento
\documentclass[a4paper, twoside, 12pt]{article}

% Pacotes usados
\usepackage[utf8]{inputenc} % enconding de caracteres
\usepackage[brazil]{babel}  % locale pt_BR
\usepackage[lmargin=2cm, rmargin=2cm, tmargin=2cm, bmargin=2cm]{geometry} % margens da folha
\usepackage{indentfirst} % sempre indenta o primeiro parágrafo

% Multicolunas
\usepackage{multicol}

% Tabelas
\usepackage{booktabs}
\usepackage{multirow}
\usepackage[table,xcdraw]{xcolor}

% Matemática
\usepackage{amsmath}
\usepackage{amsfonts}
\usepackage{gensymb}

% Tabela longa que quebra entre as paginas
\usepackage{longtable}

% Para links e url
\usepackage[hidelinks]{hyperref}

% Código fonte
\usepackage{listings} % listagem de código-fonte
\renewcommand*{\lstlistingname}{Listagem} % texto para listagem de código
\usepackage{color} % cor para usar na listagem de código-fonte

% Imagens
%\usepackage{svg}
\usepackage{graphicx,xcolor} % para inserir imagens
\usepackage[nottoc,notlot,notlof]{tocbibind} % adiciona o tópico Referências ao Sumário
\usepackage{textcomp} % accesso \textquotesingle

% Para desenhar grafos
\usepackage{tikz}
\usetikzlibrary{arrows,positioning,shapes,decorations}

% Desenhar circulos
%\usepackage{tkz-euclide}
%\usetkzobj{all}

% Escrever algoritmos em pseudo-código
%\usepackage[portuguese,linesnumbered]{algorithm2e}

% Tabelas
\usepackage{booktabs}
\usepackage{caption}

% Estilos para usar nos grafos
\tikzset{
	>=stealth',
	punkt/.style={
		rectangle,
		text centered,
		inner sep=0.7em,
		draw,
		fill=blue!5
	},
	pil/.style={
		->,
		thick,
		shorten <=2pt,
		shorten >=2pt
	}
}

% Para plotar gráficos
\usepackage{pgfplots}
\pgfplotsset{width=10cm,compat=1.9}

% Para desenhar grafos
\usepackage{tikz}
\usetikzlibrary{arrows,positioning,shapes,decorations}

% Define cores para o highlight de código-fonte
\definecolor{dkgreen}{rgb}{0,0.6,0}
\definecolor{gray}{rgb}{0.5,0.5,0.5}
\definecolor{mauve}{rgb}{0.58,0,0.82}

% Define configuração para listagem de código-fonte em linguagem C
\lstset{
	frame=tb,
	language=C,
	aboveskip=2mm,
	belowskip=2mm,
	showstringspaces=false,
	columns=flexible,
	basicstyle={\small\ttfamily},
	numbers=none,
	keywordstyle=\color{blue},
	commentstyle=\color{dkgreen},
	stringstyle=\color{mauve},
	breaklines=true,
	breakatwhitespace=false,
	tabsize=4
}

% =========================================================
%  Documento
% =========================================================

% Começo do documento
\begin{document}

% Define algum espaçamento que eu não lembro, hehe :)
\setlength\parskip{0.3cm}

% =========================================================
%  Capa
% =========================================================

% Começo da folha de Capa
\begin{titlepage}

		% Título
		\title{
\textsc {\large{Universidade de São Paulo\\
Instituto de Ciências Matemáticas e de Computação}}\\[1cm]
\large{SCC0272 -- Introdução à Computação Bioinspirada}\\[6cm]
\LARGE{Trabalho 3 -- Algoritmo ACO para resolver TSP}\\[5.5cm]
		}

		% Autores
		\author{
Elias Italiano Rodrigues -- 7987251\\
Vinicius Katata Biondo -- 6783972
		}

		% Inserção manual de data
		\date{
\vfill São Carlos, 03 de julho de 2015
		}

		% Cria a Capa
		\maketitle
		\thispagestyle{empty}

% Fim da folha de Capa
\end{titlepage}
	
% Reseta contador de página para 1 (assim não conta a Capa como página)
\setcounter{page}{1}

% =========================================================
%  Sumário
% =========================================================

% Cria o Sumário
\tableofcontents

% Cria uma nova página, forçando o Sumário a ficar numa página separada
\clearpage

% =========================================================
%  Conteúdo
% =========================================================

\section{Introdução \label{sec:introducao}}

Este trabalho implementa o algoritmo ACO (\textit{Ant Colony Optimization}) para resolver TSP (\textit{Travelling Salesman Problem}), também conhecido em português como ``O Problema do Caixeiro Viajante".

O algoritmo implementado pode ser executado com uma quantidade arbitrária de \textit{threads}. Os parâmetros do algoritmo também podem ser definidos pelo usuário como descrito na Seção~\ref{sec:params}.

O objetivo desse relatório é mostrar e comentar a execução do algoritmo para diferentes valores de \textit{formigas} e \textit{threads} para um escolhido caso de teste.

\section{O Programa \texttt{aco-tsp} \label{sec:prog}}

O programa implementa o algoritmo ACO em linguagem C especificamente para resolver TSPs do tipo \texttt{EUC\char`_2D} obtidos do TSPLIB \cite{bib:tsplib}.

\subsection{Estrutura}

O diretório do projeto que acompanha este documento está organizado da seguinte maneira:

\indent\texttt{./data}\\
\indent\indent\texttt{|-- ./tsp\char`_euc-2d} : diretório com os arquivos \texttt{.tsp} para teste.\\
\indent\indent\texttt{|-- README} : informação sobre como foram obtidos e tratados os casos de teste.\\
\indent\indent\texttt{|-- optimal.result} : arquivo com os resultados ótimos conhecidos.\\
\indent\texttt{./doc} : arquivos-fonte deste relatório.\\
\indent\texttt{Makefile} : arquivo para automação do processo de compilação do programa.\\
\indent\texttt{README} : instruções de como compilar e executar o programa.\\
\indent\texttt{RELATORIO\char`_pt-br.pdf} : este relatório compilado a partir de \texttt{./doc}.\\
\indent\texttt{aco-tsp} : programa compilado em sistema operacional Linux.\\
\indent\texttt{main.c} : código-fonte do programa em linguagem C.\\
\indent\texttt{test.sh} : \textit{shell-script} para testar o programa compilado.

\subsection{Parâmetros \label{sec:params}}

O programa foi feito para aceitar \textit{flags} pelas quais podem ser informados parâmetros de execução. Seguem as descrições para cada \textit{flag} disponível (pode-se consultá-las executando \texttt{./aco-tsp --help} na linha de comando):

\begin{verbatim}
  -i, --in           input .tsp EUC_2D file from TSPLIB
                     (default ./data/tsp_euc-2d/a280.tsp)
  -e, --expected     expected results (default ./data/optimal.result)
  -t, --threads      number of threads to be used (default 1)
  -k, --ants         number of ants (default 10)
  -a, --alpha        alpha constant (default 1.000000)
  -b, --beta         beta constant (default 1.000000)
  -r, --rho          rho constant (default 0.100000)
  -p, --pheromone    initial pheromone value (default 1.000000)
  -w, --weight       weight for depositing pheromone (default 1.000000)
  -n, --times        number of runs (default 1)
  --print            print progress of best distances for each iteration
  --help             print this help information
  --version          print version number
\end{verbatim}

\textbf{Exemplo:} o seguinte comando executa o programa 100 vezes sobre o caso de teste padrão (\texttt{./data/tsp\char`_euc-2d/a280.tsp}) utilizando 2 \textit{threads}, 20 formigas e os valores 2, 2 e 0.15 para as constantes $\alpha$, $\beta$ e $\rho$ respectivamente.

\texttt{./aco-tsp -t 2 -k 20 --alpha 2 --beta 2 --rho 0.15 -n 100}

O resultado obtido é impresso na saída padrão no seguinte formato:

\begin{verbatim}input path            : ./data/tsp_euc-2d/a280.tsp
input filename        : a280.tsp
input dimension       : 280
expected results file : ./data/optimal.result

threads   : 2
ants      : 20
alpha     : 2.000000
beta      : 2.000000
rho       : 0.150000
pheromone : 1.000000
weight    : 1.000000
times     : 100

time (s)        : 12.740000
ants result     : 2778
expected result : 2579
error           : 0.077162

ants path       :
0 279 1 2 3 4 5 6 7 8 9 10 11 12 13 14 15 16 17 18 19 20 21 22 23 24 25 26
27 28 29 30 31 32 33 34 35 36 37 38 39 40 41 42 43 44 45 46 47 48 49 50 51
52 53 54 55 56 57 58 59 60 61 62 63 64 65 66 67 68 69 70 71 72 73 74 75 76
77 78 79 80 81 82 83 84 85 86 87 88 89 90 91 92 93 94 95 96 97 98 99 100 101
102 103 104 105 106 107 108 109 110 111 112 113 114 115 116 117 118 119 120
121 122 123 124 125 126 127 128 129 130 131 132 133 134 135 136 137 138 139
140 141 142 143 144 145 146 147 148 149 150 151 152 153 154 155 156 157 158
159 160 161 162 163 164 165 166 167 168 169 170 171 172 173 174 175 176 177
178 179 180 181 182 183 184 185 186 187 188 189 190 191 192 193 194 195 196
197 198 199 200 201 202 203 204 205 206 207 208 209 210 211 212 213 214 215
216 217 218 219 220 221 222 223 224 225 226 227 228 229 230 231 232 233 234
235 236 237 238 239 240 241 242 243 244 245 246 247 248 249 250 251 252 253
254 255 256 257 258 259 260 261 262 263 264 265 266 267 268 269 270 271 272
273 274 275 276 277 278\end{verbatim}

\section{Execuções e Comentários \label{sec:execucoes}}

Escolhido o caso de teste \texttt{./data/tsp\char`_euc-2d/a280.tsp}, definiu-se executar o programa 10 vezes variando-se o número de \textit{threads} em 1, 2 e 4, e o número de formigas em 10, 100, 1000. As constantes escolhidas foram $\alpha = 2$, $\beta = 2$, $\rho = 0.15$.

% Please add the following required packages to your document preamble:
% \usepackage{multirow}
% \usepackage[table,xcdraw]{xcolor}
% If you use beamer only pass "xcolor=table" option, i.e. \documentclass[xcolor=table]{beamer}
\begin{table}[h]
\centering
\caption{Resultados obtidos nas execuções. \label{tab:execucoes}}
\begin{tabular}{|c|
>{\columncolor[HTML]{EFEFEF}}l |c|c|c|}
\hline
\cellcolor[HTML]{EFEFEF} & \multicolumn{1}{c|}{\cellcolor[HTML]{EFEFEF}} & \multicolumn{3}{c|}{\cellcolor[HTML]{EFEFEF}{\bf Formigas}} \\ \cline{3-5} 
\multirow{-2}{*}{\cellcolor[HTML]{EFEFEF}{\bf Threads}} & \multicolumn{1}{c|}{\multirow{-2}{*}{\cellcolor[HTML]{EFEFEF}{\bf Dados}}} & \cellcolor[HTML]{EFEFEF}{\bf 10} & \cellcolor[HTML]{EFEFEF}{\bf 100} & \cellcolor[HTML]{EFEFEF}{\bf 1000} \\ \hline
\multicolumn{1}{|l|}{} & {\bf Esperado} & \multicolumn{3}{c|}{\cellcolor[HTML]{EFEFEF}2579} \\ \hline\hline
 & {\bf Obitdo} & 2967 & 2847 & 2784 \\ \cline{2-5} 
 & {\bf Erro} & 0.150446 & 0.103916 & 0.079488 \\ \cline{2-5} 
\multirow{-3}{*}{1} & {\bf Tempo (s)} & 0.540838 & 5.341067 & 54.789295 \\ \hline\hline
 & {\bf Obtido} & 2952 & 2916 & 2784 \\ \cline{2-5} 
 & {\bf Erro} & 0.144630 & 0.130671 & 0.079488 \\ \cline{2-5} 
\multirow{-3}{*}{2} & {\bf Tempo (s)} & 0.271209 & 2.714911 & 27.986094 \\ \hline\hline
 & {\bf Obtido} & 2991 & 2929 & 2789 \\ \cline{2-5} 
 & {\bf Erro} & 0.159752 & 0.135712 & 0.081427 \\ \cline{2-5} 
\multirow{-3}{*}{4} & {\bf Tempo (s)} & 0.317286 & 2.706635 & 27.962556 \\ \hline
\end{tabular}
\end{table}

As execuções foram realizadas no seguinte computador:

\begin{verbatim}
Hardware:
Processor: Intel Core 2 Duo CPU T6600 @ 2.20GHz (Total Cores: 2)
Motherboard: clevo M7x0S, Chipset: SiS
System Memory: 2 x 2048 MB DRAM
Graphics: SiS [SiS] 771/671 PCIE VGA Display Adapter 256MB

Software:
OS: Ubuntu 10.04, Kernel: 2.6.32-74-generic-pae (i686)
Compiler: GCC 4.4.3
\end{verbatim}

\begin{figure}[ht]
\centering
\begin{tikzpicture}
\begin{axis}[
    xlabel={Quantidade de formigas (escala logarítmica base 10)},
    ylabel={Erro obtido},
    xmax=2000,
    ymin=0, ymax=0.2,
    xtick={0,10,100,1000},
    ytick={0,0.05,0.1,0.15,0.2},
	xmode=log,
	log basis x={10},
    scaled ticks=false,
    tick label style={/pgf/number format/fixed},
    legend pos=north east,
    ymajorgrids=true,
	xmajorgrids=true,
    grid style=dashed,
	/pgf/number format/1000 sep={},
]
\addplot[color=red,mark=square]
    coordinates {
    	(10,   0.150446)
    	(100,  0.103916)
    	(1000, 0.079488)
    };
\addlegendentry{1 \textit{thread}}

\addplot[color=blue,mark=square]
    coordinates {
    	(10,  0.144630)
    	(100, 0.130671)
    	(1000,0.079488)
    };
\addlegendentry{2 \textit{threads}}

\addplot[color=green,mark=square]
    coordinates {
    	(10,  0.159752)
    	(100, 0.135712)
    	(1000,0.081427)
    };
\addlegendentry{4 \textit{threads}}

\end{axis}
\end{tikzpicture}
\caption{Comparativo dos erros obtidos em relação ao número de formigas. \label{fig:graf1}}
\end{figure}

\begin{figure}[ht]
\centering
\begin{tikzpicture}
\begin{axis}[
    xlabel={Quantidade de threads (escala logarítmica base 2)},
    ylabel={Tempo (s)},
    xmax=5,
    ymin=0, ymax=60,
    xtick={0,1,2,4},
    ytick={0,10,20,30,40,50,60},
	xmode=log,
	log basis x={2},
    scaled ticks=false,
    tick label style={/pgf/number format/fixed},
    legend pos=north east,
    ymajorgrids=true,
	xmajorgrids=true,
    grid style=dashed,
	/pgf/number format/1000 sep={},
]
\addplot[color=red,mark=square]
    coordinates {
    	(1, 0.540838)
    	(2, 0.271209)
    	(4, 0.317286)
    };
\addlegendentry{10 formigas}

\addplot[color=blue,mark=square]
    coordinates {
    	(1, 5.341067)
    	(2, 2.714911)
    	(4, 2.706635)
    };
\addlegendentry{100 formigas}

\addplot[color=green,mark=square]
    coordinates {
    	(1, 54.789295)
    	(2, 27.986094)
    	(4, 27.962556)
    };
\addlegendentry{200 formigas}

\end{axis}
\end{tikzpicture}
\caption{Comparativo dos tempos obtidos de acordo com o número de \textit{threads}. \label{fig:graf2}}
\end{figure}

Como pode-se observar pelo gráfico da Figura~\ref{fig:graf1}, independentemente do número de \textit{threads}, o erro obtido diminuiu conforme aumentou-se o número de formigas. Esse comportamento é esperado, pois sabe-se do algoritmo que quanto maior o número de formigas, maiores são as chances de um menor caminho ser percorrido.

Pelo gráfico da Figura~\ref{fig:graf2} pode-se concluir que a quantidade de \textit{threads} influenciou diretamente no tempo de execução, pois tornou possível formigas percorrerem o caminho paralelamente. Percebe-se que o ganho de 2 para 4 \textit{threads} é muito pequeno, justificado pelas configurações do computador usado que não possui mais do que 2 \textit{cores} de processamento.

Da Tabela~\ref{tab:execucoes}, observa-se que o número de \textit{threads} não influenciou na qualidade da solução obtida \underline{nesse experimento}, pois o erro pouco variou com o aumento do número de \textit{threads}. Portanto, o aumento das \textit{threads} só permitiu uma execução mais rápida.

\clearpage
\section{Conclusão \label{sec:conclusao}}

Com este trabalho foi possível aprender mais sobre a algoritmo ACO, analisar seu comportamento na prática com a variação de parâmetros, conhecer a base de casos de teste do TSPLIB e ter uma experiência \textit{hands on} na programação do ACO em linguagem C com pthreads.


% =========================================================
%  Referências
% =========================================================

% Começo das Referências
\begin{thebibliography}{9}

	\bibitem{bib:tsplib}
		TSPLIB\\
		\textless http://www.iwr.uni-heidelberg.de/groups/comopt/software/TSPLIB95/\textgreater

% Fim das Referências
\end{thebibliography}

% Fim do documento
\end{document}
