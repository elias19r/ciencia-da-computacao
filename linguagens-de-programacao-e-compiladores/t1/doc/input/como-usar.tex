\section{Como Usar \label{sec:como-usar}}

\subsection{Compilação}

O trabalho entregue, como requisitado, já foi previamente compilado (\texttt{Linux}), não havendo necessidade de executar esse passo. Porém, caso queira ou precise compilar novamente, basta estar dentro do diretório do trabalho e executar:

	\indent\indent\texttt{make}

É necessário ter instalado o compilador \texttt{gcc}, a ferramenta \texttt{flex}, assim como o utilitário \texttt{make} em sistema operacional \texttt{Linux}.

\subsection{Execução}

Para executar o trabalho, basta estar dentro de seu diretório e executar:

	\indent\indent\texttt{./bin/main}

Dessa maneira, o programa \texttt{LALG} será lido da entrada padrão \texttt{stdin}.

Para executá-lo sobre um arquivo, basta redirecionar a entrada:

	\indent\indent\texttt{./bin/main < meu-programa.lalg}

No diretório \texttt{./test} encontram-se alguns exemplos de programa em \texttt{LALG} para testar.\\
Por exemplo:

	\indent\indent\texttt{./bin/main < ./test/programa1.lalg}

Opcionalmente, para rodar para todos os programas \texttt{.lalg} de \texttt{./test}, execute:

	\indent\indent\texttt{make run}

As saídas serão escritas em arquivos com sufixo \texttt{\char`_out} na própria pasta \texttt{./test}.

\newpage
\subsection{Exemplo de Execução}

Comando:
	
	\indent\indent\texttt{./bin/main < ./test/programa1.lalg}

Saída:
\begin{multicols}{2}
\begin{verbatim}
 2: program - program 
 2:   nome1 - IDENT (identificador) 
 2:       ; - SEMICOLON (ponto-virgula) 
 3:     var - var 
 3:       a - IDENT (identificador) 
 3:       , - COMMA (virgula) 
 3:      a1 - IDENT (identificador) 
 3:       , - COMMA (virgula) 
 3:       b - IDENT (identificador) 
 3:       : - COLON (dois-pontos) 
 3: integer - integer 
 3:       ; - SEMICOLON (ponto-virgula) 
 4:   begin - begin 
 5:    read - read 
 5:       ( - OPAR (abre parenteses) 
 5:       a - IDENT (identificador) 
 5:       ) - CPAR (fecha parenteses) 
 5:       ; - SEMICOLON (ponto-virgula) 
 6:      a1 - IDENT (identificador) 
 6:      := - ATR (atribuicao) 
 6:       a - IDENT (identificador) 
 6:       * - MULT (multiplicacao) 
 6:       2 - INTEGER (numero inteiro) 
 6:       ; - SEMICOLON (ponto-virgula) 
 8:   while - while 
 8:       ( - OPAR (abre parenteses) 
 8:      a1 - IDENT (identificador) 
 8:       > - GR (maior) 
 8:       0 - INTEGER (numero inteiro) 
 8:       ) - CPAR (fecha parenteses) 
 8:      do - do 
 9:   begin - begin 
10:   write - write 
10:       ( - OPAR (abre parenteses) 
10:      a1 - IDENT (identificador) 
10:       ) - CPAR (fecha parenteses) 
10:       ; - SEMICOLON (ponto-virgula) 
11:      a1 - IDENT (identificador) 
11:      := - ATR (atribuicao) 
11:      a1 - IDENT (identificador) 
11:       - - MINUS (subtracao) 
11:       1 - INTEGER (numero inteiro) 
11:       ; - SEMICOLON (ponto-virgula) 
12:     end - end 
12:       ; - SEMICOLON (ponto-virgula) 
14:     for - for 
14:       b - IDENT (identificador) 
14:      := - ATR (atribuicao) 
14:       1 - INTEGER (numero inteiro) 
14:      to - to 
14:      10 - INTEGER (numero inteiro) 
14:      do - do 
15:   begin - begin 
16:       b - IDENT (identificador) 
16:      := - ATR (atribuicao) 
16:       b - IDENT (identificador) 
16:       + - PLUS (adicao) 
16:       2 - INTEGER (numero inteiro) 
16:       ; - SEMICOLON (ponto-virgula) 
17:       a - IDENT (identificador) 
17:      := - ATR (atribuicao) 
17:       a - IDENT (identificador) 
17:       - - MINUS (subtracao) 
17:       1 - INTEGER (numero inteiro) 
17:       ; - SEMICOLON (ponto-virgula) 
18:     end - end 
18:       ; - SEMICOLON (ponto-virgula) 
20:      if - if 
20:       a - IDENT (identificador) 
20:      <> - DIFFERENT (diferente) 
20:       b - IDENT (identificador) 
20:    then - then 
21:   write - write 
21:       ( - OPAR (abre parenteses) 
21:       a - IDENT (identificador) 
21:       ) - CPAR (fecha parenteses) 
21:       ; - SEMICOLON (ponto-virgula) 
22:     end - end 
22:       . - DOT (ponto)
\end{verbatim}
\end{multicols}

Mais exemplos estão disponíves no diretório \texttt{./test}.
