\section{Como Usar \label{sec:como-usar}}

\subsection{Compilação}

O trabalho entregue, como requisitado, já foi previamente compilado (\texttt{Linux}), não havendo necessidade de executar esse passo. Porém, caso queira ou precise compilar novamente, basta estar dentro do diretório do trabalho e executar:

	\indent\indent\texttt{make}

É necessário ter instalado o compilador \texttt{gcc}, as ferramentas \texttt{flex} e \texttt{bison}, assim como o utilitário \texttt{make} em sistema operacional \texttt{Linux}.

\subsection{Execução}

Para executar o trabalho, basta estar dentro de seu diretório e executar:

	\indent\indent\texttt{./main}

Dessa maneira, o programa \texttt{LALG} será lido da entrada padrão \texttt{stdin}.

Para executá-lo sobre um arquivo, basta redirecionar a entrada:

	\indent\indent\texttt{./main < meu-programa.lalg}

No diretório \texttt{./test/sin} encontram-se alguns exemplos \textbf{sintáticos} de programa em \texttt{LALG} para testar. Por exemplo:

	\indent\indent\texttt{./main < ./test/sin/programa1.lalg}

Opcionalmente, para rodar para todos os programas \texttt{.lalg} de \texttt{./test/sin}, execute:

	\indent\indent\texttt{make run}

As saídas serão escritas em arquivos com sufixo \texttt{\char`_out} na própria pasta \texttt{./test/sin}.

\newpage
\subsection{Exemplo de Execução}

Arquivo \texttt{./test/sin/error\char`_varios1.lalg} -- programa fictício com vários erros:
\begin{verbatim} 1. program ; { esqueceu o nome do programa }
 2.    const c 10;  { esqueceu '=' }
 3.    var i, a, b d: integer; { esqueceu uma ',' nas variaveis }
 4. 
 5.    procedure meu_proc(v: integer; x); { esqueceu tipo do segundo parametro }
 6.    var contador: string; { tipo de dado inexistente }
 7.    begin
 8.       contador := contador + * v; { usou '*' de modo errado }
 9.       write(contador);
10.       write(x);
11.    end;
12. 
13.    function minha_func(v: integer, x: integer): integer; { usou ',' }
14. 
15. begin
16.    read(a);
17.    read(b);
18.    a = c; { fez atribuicao com sinal de '=' }
19.    read d); { esqueceu '(' }
20. 
21.    if b < then { esqueceu valor na condicao }
22.       write(b);
23. 
24.    for i := 10 to 20 { esqueceu 'do' }
25.    begin
26.       minha_func(a, d);
27.    end;
28. end { esqueceu '.' }\end{verbatim}

Comando:

	\indent\indent\texttt{./main < ./test/sin/error\char`_varios1.lalg}

Saída:
\begin{verbatim}[ 1,9 ]: syntax error, unexpected ;, expecting identificador
[ 2,10]: syntax error, unexpected valor inteiro, expecting =
[ 3,14]: syntax error, unexpected identificador, expecting :
[ 5,34]: syntax error, unexpected ), expecting :
[ 6,16]: syntax error, unexpected identificador, expecting char or
         integer or real
[ 8,26]: syntax error, unexpected *, expecting ( or identificador or
         valor inteiro or valor real
[13,32]: syntax error, unexpected ",", expecting )
[18,4 ]: syntax error, unexpected =, expecting else or ; or ( or :=
[18,6 ]: syntax error, unexpected identificador
[19,7 ]: syntax error, unexpected identificador, expecting (
[21,9 ]: syntax error, unexpected then, expecting ( or identificador or
         valor inteiro or valor real
[25,2 ]: syntax error, unexpected begin, expecting do
[26,15]: syntax error, unexpected ",", expecting ; or )
[28,5 ]: syntax error, unexpected $end, expecting .\end{verbatim}

Onde \texttt{[i,j]} indica linha \texttt{i} na coluna {\texttt{j}}.

Mais exemplos estão disponíves no diretório \texttt{./test/sin}.
