\section{Parte 1}

\subsection{A Linguagem AWK}

AWK é uma linguagem de programação interpretada para processamento de texto
comumente usada para extração de dados em documentos estruturados em \emph{registro} e \emph{campo}.

Um registro é qualquer quantidade de informação que represente uma entidade
e um campo é uma parte constituinte dessa informação.
Por exemplo: um arquivo de texto onde cada linha contenha nomes completos de alunos (registro), e
os nomes e sobrenomes (campos) estão separados por espaço. Ou ainda, mais comum, um arquivo no
formato \emph{.cvs} (Comma-separated values).

Com AWK é possível manipular tais tipos de arquivos para gerar uma nova apresentação ou fazer
alterações sistemáticas nos dados.

\subsection{A Notação BNF}

A notação BNF foi usada para escrever a primeira gramática da linguagem.
Foram definidos os conjuntos dos terminais $V_t$ e dos não-terminais $V_n$,
assim como o conjunto de regras $P$ da gramática e o não-terminal inicial \textless program\textgreater.
Nessa notação foram usadas em geral as recursões à direita para definir as possíveis cadeias.

\subsection{Definição Formal da Gramática em Notação BNF}

\lstinputlisting[language={}]{../grammar/awk.bnf}
