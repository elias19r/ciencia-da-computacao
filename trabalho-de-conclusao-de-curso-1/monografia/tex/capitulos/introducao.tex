\chapter{Introdução}
\label{ch:introducao}

\section{Contextualização e Motivação}

Este trabalho está relacionado com a área de Criptomoeda que por sua vez relaciona-se com diferentes áreas da Computação e da Economia: Sistemas Distribuídos, Redes de Computadores, Criptografia e Escola Austríaca. A motivação para sua realização vem da recente tecnologia de criptomoeda desenvolvida para o Bitcoin, colocada oficialmente em operação em janeiro de 2009 e que, desde então, tem sido pesquisada e desenvolvida. Essa tecnologia promove a descentralização e desestatização da moeda e demonstra relevante potencial para a descentralização de outros produtos e serviços.

\section{Objetivos}

Este trabalho tem como objetivo apresentar um estudo sobre Bitcoin com enfoque nas tecnologias, em particular o problema da escalabilidade da blockchain, e brevemente sobre a escola de pensamento econômico que o sustenta. Visa também contribuir como material em português para disseminação de informação sobre criptomoeda no Brasil.

\section{Organização}

O desenvolvimento deste documento está organizado da seguinte maneira. O Capítulo~\ref{ch:bitcoin-intro} apresenta a principal bibliografia deste trabalho e uma visão geral do Bitcoin, seu cenário atual, seu desenvolvimento e também conceitos de valor e moeda. O Capítulo~\ref{ch:bitcoin-tech} mostra um resumo dos principais componentes de uma criptomoeda para servir de introdução ao Capítulo~\ref{ch:escalabilidade}, que trata do problema atual de escalabilidade da blockchain. Por fim, são feitas conclusões sobre o estudo no Capítulo~\ref{ch:conclusao}.