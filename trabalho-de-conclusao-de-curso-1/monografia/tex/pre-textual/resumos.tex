\textoresumo{O surgimento do Bitcoin e demais criptomoedas trouxe uma visão diferente para o atual sistema econômico. A possibilidade de criar uma moeda descentralizada sem uma autoridade central para sua emissão e nem para pagamentos, com qualidades de uma moeda segundo a teoria da Escola Austríaca, nunca antes foi possível. Baseada na Internet, as criptomoedas funcionam por meio de um sistema distribuído em que os nós contribuem para manter o histórico das transações (blockchain) por meio de uma atividade conhecida como mineração. O sistema usa de incentivos para que os nós trabalhem honestamente e a rede é segura desde que a maioria deles sejam honestos. Porém, há um fator que nos impede de depender somente das criptomoedas como forma de dinheiro: a escalabilidade da blockchain. Este trabalho tem como objetivo apresentar uma introdução ao assunto e analisar o problema de escalabilidade.}{criptomoeda, bitcoin, blockchain, sistemas distribuídos, libertarianismo}

\textoresumo[english]{The rise of Bitcoin and other cryptocurrencies brought a different view to the current economic system. The possibility of creating a decentralized currency with no central authority to issue nor for payments, with qualities of a currency according to the Austrian School theory, has never been possible before. Based on the Internet, cryptocurrency works through a distributed system in which nodes contribute to keep the history of transactions (blockchain) by means of an activity known as mining. The system uses incentives so that nodes work honestly and the network is safe as long as most of them are honest. But there is a factor that prevents us from depending only on cryptocurrency as a form of money: the scalability of blockchain. This paper aims to present an introduction to the subject and analyze the scalability problem.}{cryptocurrency, bitcoin, blockchain, distributed systems, libertarianism}
